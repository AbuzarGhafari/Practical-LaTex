\documentclass{article}

\usepackage{commath}
\usepackage{amsmath}
\usepackage{geometry}
\usepackage{lipsum}

\setlength\parindent{0pt}


\begin{document}
	
	\begin{center}
		\section*{WPG}
		{\Large Practical \LaTeX\, Tutorial \# 22}
	\end{center}	
	\subsection*{Objectives}
	\begin{itemize}
		\item Aligned Formulas
		\item Aligned Environment
		\item Simple Alignment
		\item Annotated Alignment
		\item Annotated Alignment Integration Example
	\end{itemize}
	\hrule
	\bigskip
	
	
	\subsection*{Simple Alignment}	
	\par Align environment with each equation numbered.
	
	\begin{align}
		x_1 - 2x_2 - x_3 + 3x_4 &= 0 \\
		-2x_1 + 4x_2 + 5x_3 - 5x_4 &= 3 \\
		3x_1 - 6x_2 -6x_3 + 8x_4 &= 2
	\end{align}
		
	\par Align environment to hide equation numbers with the \textit{\textbackslash notag} command.
	\begin{align}
		x_1 - 2x_2 - x_3 + 3x_4 &= 0 \\
		-2x_1 + 4x_2 + 5x_3 - 5x_4 &= 3 \notag\\
		3x_1 - 6x_2 -6x_3 + 8x_4 &= 2 \notag
	\end{align}
	
	\par Hidding all the equation numbers by \textit{align*} environment.
	\begin{align*}
		x_1 - 2x_2 - x_3 + 3x_4 &= 0 \\
		-2x_1 + 4x_2 + 5x_3 - 5x_4 &= 3 \\
		3x_1 - 6x_2 -6x_3 + 8x_4 &= 2 
	\end{align*}
	
	
	\subsection*{Annotated Alignment}
	\begin{align*}
		\lim_{x\to c}(x^3 + 4x^2 - 3) &= \lim_{x\to c}x^3 + \lim_{x\to c}4x^2 - \lim_{x\to c} 3  && \text{Sum and Difference Rules}\\
			&= c^3 + 4c^2 - 2 && \text{Power and Multiple Rules}
	\end{align*}
	
	\begin{align*}
		\lim_{x\to c}(x^3 + 4x^2 - 3) &= \lim_{x\to c}x^3 + \lim_{x\to c}4x^2 - \lim_{x\to c} 3  && \text{Sum and Difference Rules}\\
		&= c^3 + 4c^2 - 2 && \text{\parbox{3cm}{Power and Multiple Rules}}
	\end{align*}


	\subsection*{Annotated Alignment $\vert$ Integration Exmaple}


	\begin{align*}
		\int_{0}^{\pi/4} \frac{\dif x}{1-\sin{x}} &= \int_{0}^{\pi/4} \frac{1}{1-\sin{x}} \cdot \frac{1+\sin{x}}{1+\sin{x}} \dif x && \text{\parbox{3cm}{Multiply and divide by conjugate.}} \\
		&= \int_{0}^{\pi/4} \frac{1+\sin{x}}{1-\sin^{x}} \dif x && \text{Simplify.} \\
		&= \inf_{0}^{\pi/4} \frac{1+\sin{x}}{\cos^2{x}} \dif x && \text{\parbox{3cm}{$1-\sin^2{x} = \cos^2{x}$}} \\
		&= \int_{0}^{\pi/4} \left[ \sec^2{x} + \sec^2{x} \tan{x}  \right] \dif x && \text{\parbox{3cm}{Use Table 8.1, Formulas 8 and 10}} \\
		&= (\tan{x} + \sec{x})_0^{\pi/4} \\
		&= (1 + \sqrt{2} - (0+1)) \\
		&= \sqrt{2}.
	\end{align*}
	


























	
\end{document}